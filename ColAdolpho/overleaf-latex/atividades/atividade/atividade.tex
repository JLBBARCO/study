\documentclass[a4paper, 12pt]{article}
\usepackage{graphicx} % Required for inserting images
\usepackage[brazil]{babel}
\usepackage[utf8]{inputenc}
\usepackage[top=3cm, bottom=3cm, left=2.5cm, right=2.5cm]{geometry}
\usepackage[normalem]{ulem}
\usepackage{graphicx, color}
\usepackage{enumerate}
\usepackage{float}

\title{Valorant}
\author{JOSE LUIZ BRUIANI BARCO}
\date{April, 01 of 2024}

\begin{document}

\maketitle
\section{Introdução}
Valorant é um jogo de FPS (Tiro em Primeira Pessoa), misturado com magia, onde há no total dois times, o Atacante e o Defensivo, e cada time possui 05 integrantes.O lado Atacante tem o objetivo de plantar a SPIKE (bomba plantada em um retângulo, nomeado por: A, B e C), enquanto o Defensivo, como o próprio nome diz, tem que defender e impedir que a SPIKE seja plantada.

No total, são no máximo 25 partidas, mas pode ser que tenha só 19, dependendo da diferença de habilidades entre os dois times. Essas outras 06, são se acaso o jogo estiver muito equilibrado, então há a 25ª partida que é chamada de morte súbita, pois quem pontuar, vence.

Durante o jogo, há a troca de posições, onde o Atacante vira Defensivo, e vice-versa. Essa troca acontece, normalmente, entre a 11ª e 12ª partida, onde quem era Atacante, até a 11ª partida, na 12ª, vira Defensor.

Há uma variedade grande de personagens, cada um com uma habilidade diferente, mas o meu preferido é o SOVA, onde a habilidade dele é de Sniper, e prefiro jogar de Sniper do que rushador.
\section{Top-10 personagens mais populares}
\begin{enumerate}
    \item Reyna
    \begin{figure}[H]
        \centering
    \end{figure}
    \begin{tabular}{c}
    \includegraphics[width=0.5\textwidth]
    {Reyna.png}
    \end{tabular}
    \item Jett
    \begin{figure}[H]
        \centering
    \end{figure}
    \begin{tabular}{c}
    \includegraphics[width=0.5\textwidth]
    {Jett.png}
    \end{tabular}
    \item Sage
    \begin{figure}[H]
        \centering
    \end{figure}
    \begin{tabular}{c}
    \includegraphics[width=0.5\textwidth]
    {Sage.png}
    \end{tabular}
    \item Omen
    \begin{figure}[H]
        \centering
    \end{figure}
    \begin{tabular}{c}
    \includegraphics[width=0.5\textwidth]
    {Omen.png}
    \end{tabular}
    \item Raze
    \begin{figure}[H]
        \centering
    \end{figure}
    \begin{tabular}{c}
    \includegraphics[width=0.5\textwidth]
    {Raze.png}
    \end{tabular}
    \item Killjoy
    \begin{figure}[H]
        \centering
    \end{figure}
    \begin{tabular}{c}
    \includegraphics[width=0.5\textwidth]
    {Killjoy.png}
    \end{tabular}
    \item Brimstone
    \begin{figure}[H]
        \centering
    \end{figure}
    \begin{tabular}{c}
    \includegraphics[width=0.5\textwidth]
    {Brimstone.png}
    \end{tabular}
    \item Skye
    \begin{figure}[H]
        \centering
    \end{figure}
    \begin{tabular}{c}
    \includegraphics[width=0.5\textwidth]
    {Skye.png}
    \end{tabular}
    \item Sova
    \begin{figure}[H]
        \centering
    \end{figure}
    \begin{tabular}{c}
    \includegraphics[width=0.5\textwidth]
    {Sova.png}
    \end{tabular}
    \item Phoenix
    \begin{figure}[H]
        \centering
    \end{figure}
    \begin{tabular}{c}
    \includegraphics[width=0.5\textwidth]
    {Phoenix.png}
    \end{tabular}
\end{enumerate}
\end{document}
